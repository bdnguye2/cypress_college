\documentclass[11pt]{beamer}

\usetheme{metropolis}

\usepackage{graphicx}
\usepackage{physics}
\usepackage{adjustbox}
\usepackage{caption}
\usepackage{chemformula}
\usepackage{quoting}
\usepackage[style=chem-angew,backend=bibtex]{biblatex}
\bibliography{references}
%
% Choose how your presentation looks.
%
% For more themes, color themes and font themes, see:
% http://deic.uab.es/~iblanes/beamer_gallery/index_by_theme.html
%
\mode<presentation>
{
  \usetheme{default}      % or try Darmstadt, Madrid, Warsaw, ...
  \usecolortheme{default} % or try albatross, beaver, crane, ...
  \usefonttheme{default}  % or try serif, structurebold, ...
  \setbeamertemplate{navigation symbols}{}
  \setbeamertemplate{caption}[numbered]
  \setbeamerfont{footnote}{size=\tiny}
} 

\usepackage[english]{babel}
\usepackage[utf8]{inputenc}
\graphicspath{{image/}}

\AtBeginSection[]{
\begin{frame}{Outline}
  \tableofcontents[currentsection]
\end{frame}
}

\title{Chapter 1: Sharpening the Math Toolbox}
\institute{Chemistry Department, Cypress College}
\date{August 22, 2022}

\begin{document}

\begin{frame}
  \titlepage
\end{frame}

\section{Introduction: Who am I?}

\begin{frame}{Introduction: Who am I?}
  \begin{center}
    \includegraphics[angle=-90,origin=c,scale=0.03]{triple_ice}
  \end{center}
  \begin{itemize}
  \item Last June, graduated from University of California, Irvine,
    receiving my PhD in Computational and Theoretical Chemistry
  \item May refer me as Dr. Prof. Brian D. Nguyen
  \item Exercising, driving, hiking, learning new languages, and gaming
  \end{itemize}
\end{frame}

\begin{frame}{Introduction: Your Turn}
  \textbf{With your notecard:}
  \begin{itemize}
  \item Take 2-3 mins and write down your name on one side
  \item On the other side, write down something that I can
    remember you by
  \end{itemize}
\end{frame}

\section{Review: Syllabus}

\section{Math Review for Chemist}

\begin{frame}
  \vspace{1in}
  \begin{quote}
    Chemistry is necessarily an experimental science: its
    conclusions are drawn from data, and its principles
    supported by evidence from facts. - Michael Faraday
  \end{quote}
  
  \begin{center}
    \includegraphics[scale=0.08]{micahel_fara}
  \end{center}
\end{frame}

\begin{frame}{Scientific Notation}
  The scientific notation is expressed
  \begin{equation}
    N = C \times 10^m
  \end{equation}
  where $N$ is a large number, $C$ is the coefficient (a number between $1-9$)
  and $m$ is the exponent (a positive or negative integer)

  \textbf{Example:} $0.00363246 = 3.63246 \times 10^{-3}$

\end{frame}

\begin{frame}{Significant Figures}
  \begin{itemize}
  \item The meaningful digits in a measured or calculated
    quantity
  \item Example: $0.00363246 \simeq 3.63 \times 10^{-3}$ to three
    sig figures
  \item Implies relative accuracy of $10^{-m}$, e.g. $0.1\%$ for $m=3$
  \item[] \begin{center}\includegraphics[scale=0.1]{grad}
  \end{center}
  \item For practice, what is the measured volume for the liquid
    above?
  \end{itemize}
\end{frame}

\begin{frame}{Significant Figures - More Practice!}
  \centering
  \includegraphics[scale=0.22]{graph_measure}

  \textbf{Which is relatively more accurate? What is the
    approximate measurement for each graph? What is missing?}
\end{frame}

\begin{frame}{Counting Significant Figures}
  All \textbf{non-zero} numbers in a measured number are
  significant

  \textbf{Practice:} What is the number of significant figures?
  \begin{itemize}
  \item 36.1 ft
  \item 1 dozen eggs
  \item 155.6 lbs
  \end{itemize}
\end{frame}

\begin{frame}{Leading, Sandwiched and Trailing Zeroes}
  \textbf{Leading zeroes:} Precede non-zero digits in a
  decimal number are \textbf{not} significant

  \textbf{Sandwiched zeroes:} Occur between nonzero numbers are significant
  
  \textbf{Trailing zeroes:} Following non-zero numbers are
  significant in numbers with a decimal point
\end{frame}

\begin{frame}{Leading, Sandwiched and Trailing Zeroes}
  \textbf{Practice:} What is the number of significant figures?
  \begin{itemize}
  \item 0.0702 lb
  \item 48600 L
  \item 100.000 g
  \item 1.020 atm
  \item $9.01 \times 10^5$ m
  \end{itemize}
\end{frame}

\begin{frame}{Calculated Answers}
  \centering
  \includegraphics[scale=0.2]{calc}
  \begin{itemize}
  \item Answers must have the same number of significant
    figures as the least precise measured number(s)
  \item Calculator answers must often be \textbf{rounded off}
  \item \textbf{Rounding rules} are used to obtain the correct
    number of significant figures
  \end{itemize}
\end{frame}

\begin{frame}{TIPS: Avoid Rounding Errors}
  \begin{itemize}
  \item Carry at least 2 extra significant figures in
    intermediate results!
  \item You will need to report your results \emph{exactly} to a
    given precision
  \item Round at the very end
  \end{itemize}

  \textbf{Practice:} Round to four significant figures.
  \begin{itemize}
  \item 824.75143 cm
  \item 0.112544 g
  \end{itemize}
\end{frame}

\begin{frame}{}
\end{frame}

\begin{frame}{}
\end{frame}

\begin{frame}{Accuracy vs. Precision}
  \textbf{What is the difference between accuracy
    and precision?}

  \centering
  \includegraphics[scale=0.15]{accur_prec.png}
\end{frame}

\begin{frame}{Accuracy vs. Precision}
  \textbf{Accuracy}
  \begin{itemize}
  \item How close you are to the actual value
  \item Calculated by the forumula
  \item[] \begin{equation}
    \% \text{Error} = \frac{\text{measured} - \text{actual}}{\text{actual}}
  \end{equation}
  \end{itemize}
  
  \textbf{Precision}
  \begin{itemize}
  \item How finely tuned your measurements are or
    how close they can be to each other
  \item Depends on the measuring tool
  \item Implied by the number of significant figures
  \end{itemize}
\end{frame}


\end{document}

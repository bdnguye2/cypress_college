%%%%%%%%%%%%%%%%%%%%%%%%%%%%%%%%%%%%%%%%%%%%%%%%%%%%%%%%%%%%%%%%%%%%%%%%
% Preamble
%%%%%%%%%%%%%%%%%%%%%%%%%%%%%%%%%%%%%%%%%%%%%%%%%%%%%%%%%%%%%%%%%%%%%%%%
\documentclass[12pt]{article}
%
% Packages and other includes
% Pagination
\usepackage[letterpaper, margin=1in]{geometry}
%
% Graphics, floats, tables
\usepackage{graphicx, color, float, array}
\graphicspath{{image/}}
%
% Fonts
\usepackage[T1]{fontenc} % best for Western European languages
\usepackage{lmodern} % Latin Modern instead of CM
\usepackage{textcomp} % required to get special symbols
%
% Math
\usepackage{amsmath, amssymb}
\usepackage{enumerate}
\usepackage{braket}
% 
% Hyperlinks
\usepackage[colorlinks,linkcolor={red},citecolor={blue},
urlcolor={blue}]{hyperref} 
%
% Definitions and settings
% Paragraph indent and spacing
\setlength{\parskip}{0.4\baselineskip}
\setlength{\parindent}{0in}
%
% Math mode version of "r" column type (requires array package)
\newcolumntype{R}{>{$}r<{$}}
% Title, authors, date
\title{\textbf{Ch 7: Electromagnetic Radiation}}
\date{Nov 7, 2022}

\begin{document}

\maketitle 

\textbf{Quantum Chemistry}

1) What is the Schr\"{o}dinger cat thought experiment? How should
electrons be viewed? What are the atomic orbital describing?
\vspace{1in}

2) Why is the energy gaps between subsequent principle energy levels
$n$ become smaller e.g. $n=1$ to $n=2$, $n=2$ to $n=3$, etc.
\vspace{1in}

\textbf{Electromagnetic Radiation}

3) Which transition in the hydrogen atom results in emitted light with
the longest wavelength? Explain. Sketch the relative energy levels in the
hydrogen atom.

a) $n = 4 \rightarrow n = 3$

b) $n = 2 \rightarrow n = 1$

c) $n = 3 \rightarrow n = 2$

4) An electron in a hydrogen atom is excited with electrical energy to
an excited state with n = 2. The atom then emits a photon. What is
the value of n for the electron following the emission?
\vspace{1in}

5) Calculate the frequency and energy of each wavelength of electromagnetic
radiation. For each, what is the energy for 1 mole of photon? ($N_A = 6.022\times 10^{23}$) 

a) 632.8 nm (wavelength of red light from helium–neon laser)

b) 503 nm (wavelength of maximum solar radiation)

c) 0.052 nm (wavelength contained in medical X-rays)

\textbf{Electron Configurations and Periodic Trends}

6) Write the electron configuration for each ion:
O$^{2-}$, Br$^-$, Sr$^{2+}$, Co$^{3+}$. Cu$^{2+}$, Cl$^-$, P$^{3-}$,
K$^-$, Mo$^{3+}$, and V$^{3+}$

4) Consider these elements: N, Mg, O, F, Al.

a) Write the electron configuration for each element.

b) Arrange the elements in order of decreasing atomic radius.

c) Arrange the elements in order of increasing ionization energy.

d) Use the electron configurations in part a to explain the
differences between your answers to parts b and c.
\vspace{0.5in}

7) Explain why atomic radius decreases as we move to the right
across a period for main-group elements but not for transition
elements.
\vspace{1in}

\textbf{Taking it Further: Particle in a Box}

8) An electron confined to a one-dimensional box has energy levels
given by the equation
\begin{equation}
  E_n =\frac{n^2h^2}{8mL^2}
\end{equation}
where $n$ is a quantum number with possible values of 1, 2, 3,$\cdots$, $m$
is the mass of the electron ($9.109\times 10^{-31}$ kg), and L is the length of the box.

a) Calculate the energies of the $n = 1$, $n = 2$, and $n = 3$ levels for
an electron in a box with a length of 155 pm.

b) Calculate the wavelength of light required to make a transition
from $n = 1 \rightarrow n = 2$ and from $n = 2 \rightarrow n = 3$.
In what region of the electromagnetic radiation do these wavelength lie?

\end{document}

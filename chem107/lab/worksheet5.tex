%%%%%%%%%%%%%%%%%%%%%%%%%%%%%%%%%%%%%%%%%%%%%%%%%%%%%%%%%%%%%%%%%%%%%%%%
% Preamble
%%%%%%%%%%%%%%%%%%%%%%%%%%%%%%%%%%%%%%%%%%%%%%%%%%%%%%%%%%%%%%%%%%%%%%%%
\documentclass[12pt]{article}
%
% Packages and other includes
% Pagination
\usepackage[letterpaper, margin=1in]{geometry}
%
% Graphics, floats, tables
\usepackage{graphicx, color, float, array}
\graphicspath{{image/}}
%
% Fonts
\usepackage[T1]{fontenc} % best for Western European languages
\usepackage{lmodern} % Latin Modern instead of CM
\usepackage{textcomp} % required to get special symbols
%
% Math
\usepackage{amsmath, amssymb}
\usepackage{enumerate}
\usepackage{braket}
% 
% Hyperlinks
\usepackage[colorlinks,linkcolor={red},citecolor={blue},
urlcolor={blue}]{hyperref} 
%
% Definitions and settings
% Paragraph indent and spacing
\setlength{\parskip}{0.4\baselineskip}
\setlength{\parindent}{0in}
%
% Math mode version of "r" column type (requires array package)
\newcolumntype{R}{>{$}r<{$}}
% Title, authors, date
\title{\textbf{Ch 4: Chemical Composition}}
\date{Sept 21, 2022}

\begin{document}

\maketitle 

\textbf{Mass Percent}

1) What is the percent composition by mass of aspartame (C$_{14}$H$_{18}$N$_2$O$_5$),
the artificial sweetener NutraSweet? Report to 3 sig figs.

\vspace{0.5in}

2) Most fertilizers consist of nitrogen-containing compounds such as NH$_3$, CO(NH$_2$)$_2$,
NH$_4$NO$_3$, and (NH$_4$)$_2$SO$_4$. Plants use the nitrogen content in these compounds
for protein synthesis. Calculate the mass percent composition of nitrogen in each of the
fertilizers named in this problem. Which fertilizer has the highest nitrogen content?

\vspace{0.5in}

3) The American Dental Association recommends that an adult female should consume 3.0 mg
of fluoride (F$^-$) per day to prevent tooth decay. If the fluoride is consumed in the form
of sodium fluoride ($45.24\%$ F), what amount of sodium fluoride contains the recommended
amount of fluoride? Report to 2 sig figs.

\vspace{0.5in}


\textbf{Molar Mass}

4) Calculate the mass (in g) of each sample.

a) $5.94 \times 10^{20}$ SO$_3$ molecules

b) $2.8 \times 10^{22}$ H$_2$O molecules

c) $9.85 \times 10^{19}$ CCl$_2$F$_2$ molecules

\vspace{0.5in}

5) How many molecules of ethanol (C2H5OH) (the alcohol in alcoholic beverages) are present
in 145 mL of ethanol? The density of ethanol is 0.789 g/cm3. Report to 3 sig figs.

\vspace{0.5in}

6) A mixture of KCl and KBr has a mass of 4.00 g and contains 1.00 g of Na. What is the
mass of KCl in the mixture? Report to 3 sig figs.

\vspace{0.5in}

\textbf{Empirical and Molecular Formula}

7) A chemist decomposes samples of several compounds; the masses of their constituent
elements are shown. Calculate the empirical formula for each compound.

a) 1.651 g Ag, 0.1224 g O

b) 0.672 g Co, 0.569 g As, 0.486 g O

c) 1.443 g Se, 5.841 g Br

\vspace{0.5in}

8) Ascorbic acid (vitamin C) is $40.92\%$ C, $4.58\%$ H and $54.50\%$ O by mass.
What is the empirical formula of ascorbic acid?

\vspace{0.5in}

9) Estradiol is a female sexual hormone that causes maturation and maintenance of the female
reproductive system. Elemental analysis of estradiol gives the following mass percent composition:
C $79.37\%$, H $8.88\%$, O $11.75\%$. The molar mass of estradiol is 272.37 g/mol. Find the
molecular formula of estradiol.

\vspace{0.5in}

\textbf{Molarity and Dilution}

10) What volume of 0.200 M ethanol solution contains each of the following amounts?
Report to 3 sig figs.

a) $0.450$ mol ethanol

b) $1.22$ mol ethanol

c) $1.20 \times 10^{-2}$ mol ethanol

\vspace{0.5in}

11) If you dilute 175 mL of a 1.6 M solution of LiCl to 0.5 M, determine the new volume of the
solution. Report to 2 sig figs.

\vspace{1in}

12) To make 10.0 L of 1.25 M KNO$_3$, what molarity would the potassium nitrate solution
need to be if you were to use only 2.5 L of the more concentrated KNO$_3$? Report to 3 sig figs.

\vspace{0.5in}

13) To what volume should you dilute 100. mL of an 7.90 M CuCl$_2$ solution so that 51.5 mL of
the diluted solution contains 4.49 g CuCl$_2$? Report to 3 sig figs.

\end{document}

%%%%%%%%%%%%%%%%%%%%%%%%%%%%%%%%%%%%%%%%%%%%%%%%%%%%%%%%%%%%%%%%%%%%%%%%
% Preamble
%%%%%%%%%%%%%%%%%%%%%%%%%%%%%%%%%%%%%%%%%%%%%%%%%%%%%%%%%%%%%%%%%%%%%%%%
\documentclass[12pt]{article}
%
% Packages and other includes
% Pagination
\usepackage[letterpaper, margin=1in]{geometry}
%
% Graphics, floats, tables
\usepackage{graphicx, color, float, array}
\graphicspath{{image/}}
%
% Fonts
\usepackage[T1]{fontenc} % best for Western European languages
\usepackage{lmodern} % Latin Modern instead of CM
\usepackage{textcomp} % required to get special symbols
%
% Math
\usepackage{amsmath, amssymb}
\usepackage{enumerate}
\usepackage{braket}
% 
% Hyperlinks
\usepackage[colorlinks,linkcolor={red},citecolor={blue},
urlcolor={blue}]{hyperref} 
%
% Definitions and settings
% Paragraph indent and spacing
\setlength{\parskip}{0.4\baselineskip}
\setlength{\parindent}{0in}
%
% Math mode version of "r" column type (requires array package)
\newcolumntype{R}{>{$}r<{$}}
% Title, authors, date
\title{\textbf{Energy Changes}}
\date{Oct 10, 2022}

\begin{document}

\maketitle 

\textbf{Conservation of Energy}

1) We submerge a silver block, initially at 58.5°C, into 100.0 g of
water at 24.8°C, in an insulated container. The final temperature
of the mixture upon reaching thermal equilibrium is 26.2°C. What is
the mass of the silver block?

\textbf{Bomb Calorimetry}

2) When 0.514 g of biphenyl (C$_{12}$H$_{10}$) undergoes combustion in a
bomb calorimeter, the temperature rises from 25.8°C to 29.4°C.
Write the balanced combustion reaction equation of biphenyl.
Find the amount of heat released from the combustion of biphenyl.
The heat capacity of the bomb calorimeter, determined in a separate
experiment, is 5.86 kJ/°C.

3) When 1.010g sucrose (C$_{12}$H$_{22}$O$_{11}$) undergoes combustion in a bomb
calorimeter, the temperature rises from 24.92°C to 28.33°C. Write the balanced
combustion reaction equation of sucrose. Find $\Delta E_\text{rxn}$ for the combustion
of sucrose in kJ/mol sucrose. The heat capacity of the bomb calorimeter, determined
in a separate experiment, is 4.90 kJ/°C. (You can ignore the heat capacity of the
small sample of sucrose because it is negligible compared to the heat capacity of
the calorimeter.)

\textbf{Bring It Together}

4) A 25.5g aluminum block is warmed to 65.4°C and plunged into an insulated beaker
containing 55.2g water initially at 22.2°C. The aluminum and the water are allowed
to come to thermal equilibrium. Assuming that no heat is lost, what is the final
temperature of the water and aluminum?

\end{document}

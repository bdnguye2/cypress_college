%%%%%%%%%%%%%%%%%%%%%%%%%%%%%%%%%%%%%%%%%%%%%%%%%%%%%%%%%%%%%%%%%%%%%%%%
% Preamble
%%%%%%%%%%%%%%%%%%%%%%%%%%%%%%%%%%%%%%%%%%%%%%%%%%%%%%%%%%%%%%%%%%%%%%%%
\documentclass[12pt]{article}
%
% Packages and other includes
% Pagination
\usepackage[letterpaper, margin=1in]{geometry}
%
% Graphics, floats, tables
\usepackage{graphicx, color, float, array}
\graphicspath{{image/}}
%
% Fonts
\usepackage[T1]{fontenc} % best for Western European languages
\usepackage{lmodern} % Latin Modern instead of CM
\usepackage{textcomp} % required to get special symbols
%
% Math
\usepackage{amsmath, amssymb}
\usepackage{enumerate}
\usepackage{braket}
% 
% Hyperlinks
\usepackage[colorlinks,linkcolor={red},citecolor={blue},
urlcolor={blue}]{hyperref} 
%
% Definitions and settings
% Paragraph indent and spacing
\setlength{\parskip}{0.4\baselineskip}
\setlength{\parindent}{0in}
%
% Math mode version of "r" column type (requires array package)
\newcolumntype{R}{>{$}r<{$}}
% Title, authors, date
\title{\textbf{Ch 8+9: Lewis and Gas Laws}}
\date{\today}

\begin{document}

\maketitle 

\textbf{Lewis Structures}

1) Draw the Lewis structure and determing the geometry for the following
compounds: CH$_4$, CH$_3$Cl, CH$_2$Cl$_2$,  CHCl$_3$ and CCl$_4$

a) Determine which bonds are polar.

b) Determine whether the molecule is polar.

\vspace{0.5in}

2) Draw the Lewis structure for acetic acid (CH$_2$COOH)

a) Determine which bonds are polar.

b) Determine whether the molecule is polar.

c) Draw the Lewis structure of the anion acetate (CHCOO$^-$)
and determine the formal charges on the atoms. Include all possible
resonance structures.

\vspace{0.5in}

3) Draw the Lewis structures, classify the geometry, determine which molecules
are polar, and if nonpolar, explain why: CO$_2$, SO$_2$, SO$_3$, BH$_3$, and O$_2$

\vspace{0.5in}

\textbf{Gas Laws}

4) A sample of oxygen gas occupies a volume of 250. mL at a pressure of
740. torr. What volume will the gas occupy at a pressure of 800. torr if
temperature is held constant.

\vspace{0.5in}

5) A sample of nitrogen occupies a volume of 250 mL at 25°C. What volume will
it occupy at 95°C?

\vspace{0.5in}

6) An alternate way to state Avogadro’s law is “All other things being equal,
the number of molecules in a gas is directly proportional to the volume of the gas.”

a) What is the meaning of the term “directly proportional?”

b) What are the “other things” that must be equal?

\vspace{0.5in}

7) A balloon inflated with three breaths of air has a volume of 1.7 L. At the
same temperature and pressure, what is the volume of the balloon if five more
same-sized breaths are added to the balloon?

\vspace{0.5in}

8) 5.00 L of a gas is known to contain 0.965 mol. If the amount of gas is increased
to 1.80 mol, what new volume will result (at an unchanged temperature and pressure)? 

\end{document}

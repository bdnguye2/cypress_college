%%%%%%%%%%%%%%%%%%%%%%%%%%%%%%%%%%%%%%%%%%%%%%%%%%%%%%%%%%%%%%%%%%%%%%%%
% Preamble
%%%%%%%%%%%%%%%%%%%%%%%%%%%%%%%%%%%%%%%%%%%%%%%%%%%%%%%%%%%%%%%%%%%%%%%%
\documentclass[12pt]{article}
%
% Packages and other includes
% Pagination
\usepackage[letterpaper, margin=1in]{geometry}
%
% Graphics, floats, tables
\usepackage{graphicx, color, float, array}
\graphicspath{{image/}}
%
% Fonts
\usepackage[T1]{fontenc} % best for Western European languages
\usepackage{lmodern} % Latin Modern instead of CM
\usepackage{textcomp} % required to get special symbols
%
% Math
\usepackage{amsmath, amssymb}
\usepackage{enumerate}
\usepackage{braket}
% 
% Hyperlinks
\usepackage[colorlinks,linkcolor={red},citecolor={blue},
urlcolor={blue}]{hyperref} 
%
% Definitions and settings
% Paragraph indent and spacing
\setlength{\parskip}{0.4\baselineskip}
\setlength{\parindent}{0in}
%
% Math mode version of "r" column type (requires array package)
\newcolumntype{R}{>{$}r<{$}}
% Title, authors, date
\title{\textbf{Worksheet 6}}
\date{Oct 3, 2022}

\begin{document}

\maketitle 

\textbf{Double Displacement Reactions}

1) Complete and balance each equation. If no reaction occurs,
write NO REACTION. (Optional - practice naming these compounds)

\begin{enumerate}[(a)]
\item LiI(aq) + BaS(aq) $\rightarrow$
\item KCl(aq) + CaS(aq) $\rightarrow$
\item CrBr$_2$(aq) + Na$_2$CO$_3$(aq) $\rightarrow$
\item NaOH(aq) + FeCl$_3$(aq) $\rightarrow$
\item NaNO$_3$(aq) + KCl(aq) $\rightarrow$
\item (NH$_4$)$_2$SO$_4$(aq) + SrCl$_2$(aq) $\rightarrow$
\item NH$_4$Cl(aq) + AgNO$_3$(aq) $\rightarrow$
\end{enumerate}

2) Complete and balance the acid-base reactions.

\begin{enumerate}[(a)]
\item H$_2$SO$_4$(aq) + Ca(OH)$_2$(aq) $\rightarrow$
\item HClO$_4$(aq) + KOH(aq) $\rightarrow$
\item HC$_2$H$_3$O$_2$(aq) + Ca(OH)$_2$(aq) $\rightarrow$
\item HBr(aq) + NaOH(aq) $\rightarrow$
\end{enumerate}

3) From problem 2, suppose you have 5.0g of each acid dissolved
in 100mL solvent. What volume of 0.5M of base is needed to neutralize
the acid in each chemical equation?

\newpage

\textbf{Reaction Stoichiometry}

4) Consider the unbalanced equation for the combustion of hexane
C$_6$H$_{14}$(g). Write and balance the chemical equation. Determine
how many moles of O$_2$(g) are required to react completely with
150g of hexane.

5) Suppose you want to neutralize 0.500L of 0.5M acetic
acid HC$_2$H$_3$O$_2$(aq) with Ba(OH)$_2$(aq). How many moles of
Ba(OH)$_2$(aq) is needed to neutralize the acid?

6) For each combination reaction, complete the reaction.
Calculate the mass in grams of the product
that forms when 3.67g of the underlined reactant completely reacts.
Assume that there is more than enough of the other reactant.

\begin{enumerate}[(a)]
\item \underline{Ba}(s) + Cl$_2$(g) $\rightarrow$ %BaCl$_2$(s)
\item \underline{CaO}(s) + CO$_2$(g) $\rightarrow$ %CaCO$_3$(s)
\item \underline{Mg}(s) + O$_2$(g) $\rightarrow$ %MgO(s)
\item \underline{Al}(s) + O$_2$(g) $\rightarrow$ %Al$_2$O$_3$(s)
\item K(s) + \underline{Cl$_2$}(g) $\rightarrow$ %KCl(s)
\item \underline{Sr}(s) + O$_2$(g) $\rightarrow$ %SrO(s)
\end{enumerate}

7) For each chemical reaction in problem 6, how much of the other
reactant is needed in mols to completely react with all the underlined
reactant?
  

\end{document}

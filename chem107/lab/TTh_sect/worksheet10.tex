%%%%%%%%%%%%%%%%%%%%%%%%%%%%%%%%%%%%%%%%%%%%%%%%%%%%%%%%%%%%%%%%%%%%%%%%
% Preamble
%%%%%%%%%%%%%%%%%%%%%%%%%%%%%%%%%%%%%%%%%%%%%%%%%%%%%%%%%%%%%%%%%%%%%%%%
\documentclass[12pt]{article}
%
% Packages and other includes
% Pagination
\usepackage[letterpaper, margin=1in]{geometry}
%
% Graphics, floats, tables
\usepackage{graphicx, color, float, array}
\graphicspath{{image/}}
%
% Fonts
\usepackage[T1]{fontenc} % best for Western European languages
\usepackage{lmodern} % Latin Modern instead of CM
\usepackage{textcomp} % required to get special symbols
%
% Math
\usepackage{amsmath, amssymb}
\usepackage{enumerate}
\usepackage{braket}
% 
% Hyperlinks
\usepackage[colorlinks,linkcolor={red},citecolor={blue},
urlcolor={blue}]{hyperref} 
%
% Definitions and settings
% Paragraph indent and spacing
\setlength{\parskip}{0.4\baselineskip}
\setlength{\parindent}{0in}
%
% Math mode version of "r" column type (requires array package)
\newcolumntype{R}{>{$}r<{$}}
% Title, authors, date
\title{\textbf{Ch 7: Electromagnetic Radiation}}
\date{\today}

\begin{document}

\maketitle 

\textbf{Electromagnetic Radiation}

1) An electron in a hydrogen atom is excited with electrical energy to
an excited state with n = 2. The atom then emits a photon. What is
the value of n for the electron following the emission?
\vspace{1in}

2) Calculate the frequency and energy of each wavelength of electromagnetic
radiation. For each, what is the energy for 1 mole of photon? ($N_A = 6.022\times 10^{23}$ photon/mol) 

a) 632.8 nm (wavelength of red light from helium–neon laser)

b) 503 nm (wavelength of maximum solar radiation)

c) 0.052 nm (wavelength contained in medical X-rays)

d) 2.97 m (wavelength for FM radio station)

\vspace{1in}

3) What is the frequency of light that is composed of photons that each
has a nenergy of $1.99\times 10^{-25}$ J? What type of electromagnetic
radiation is this?
\newpage

4) Which transition in the hydrogen atom results in emitted light with
the longest wavelength? Explain. Sketch the relative energy levels in the
hydrogen atom.

a) $n = 2 \rightarrow n = 1$

b) $n = 4 \rightarrow n = 3$

c) $n = 3 \rightarrow n = 2$
\vspace{1in}

\textbf{Electron Configurations}

5) Write the electron configuration for the following atoms:
As, Se, Mg, Na, Ba, Sb, I
\vspace{2in}

6) Write the electron configuration for each ion:
O$^{2-}$, Br$^-$, Sr$^{2+}$, Co$^{3+}$. Cu$^{2+}$, Cl$^-$, P$^{3-}$,
K$^-$, Mo$^{3+}$, and V$^{3+}$

\end{document}

%%%%%%%%%%%%%%%%%%%%%%%%%%%%%%%%%%%%%%%%%%%%%%%%%%%%%%%%%%%%%%%%%%%%%%%%
% Preamble
%%%%%%%%%%%%%%%%%%%%%%%%%%%%%%%%%%%%%%%%%%%%%%%%%%%%%%%%%%%%%%%%%%%%%%%%
\documentclass[12pt]{article}
%
% Packages and other includes
% Pagination
\usepackage[letterpaper, margin=1in]{geometry}
%
% Graphics, floats, tables
\usepackage{graphicx, color, float, array}
\graphicspath{{image/}}
%
% Fonts
\usepackage[T1]{fontenc} % best for Western European languages
\usepackage{lmodern} % Latin Modern instead of CM
\usepackage{textcomp} % required to get special symbols
%
% Math
\usepackage{amsmath, amssymb}
\usepackage{enumerate}
\usepackage{braket}
% 
% Hyperlinks
\usepackage[colorlinks,linkcolor={red},citecolor={blue},
urlcolor={blue}]{hyperref} 
%
% Definitions and settings
% Paragraph indent and spacing
\setlength{\parskip}{0.4\baselineskip}
\setlength{\parindent}{0in}
%
% Math mode version of "r" column type (requires array package)
\newcolumntype{R}{>{$}r<{$}}
% Title, authors, date
\title{\textbf{Ideal Gas Laws}}
\date{Nov 30, 2022}

\begin{document}

\maketitle 

\textbf{Ideal Gas Law}

1) Graph the relationship of P vs V (Boyle's Law), V vs T (Charles' Law),
and V vs n (Avogadro's hypothesis). Explain the relationship

\vspace{0.25in}

2) Assuming constant temperature. Determine the final pressure when 8.00mL
krypton at 1.97atm is transferred to a vessel of volume 1.0L.

\vspace{0.25in}

3) An outdoor storage container for hydrogen gas with a volume of 300kL is
at 2.0 atm and 10$^\circ$C. The temperature rises to 40$^\circ$C. What is the
new pressure of the hydrogen in the container?

\vspace{0.25in}

4) What is the density of chloroform (CHCl$_3$) with a vapor pressure
at 0.267atm and 300K?

\vspace{0.25in}

5) What mass of ammonia (NH$_3$) will exert the same pressure as 10mg of
hydrogen sulfide (H$_2$S) in the same container under the same conditions?

\vspace{0.25in}

6) Nitroglycerin (C$_3$H$_5$(NO$_3$)$_3$) is highly sensitive and detonates
by the reaction

4 C$_3$H$_5$(NO$_3$)$_3$(l) $\rightarrow$ 6 N$_2$(g) + 10 H$_2$O(g) + 12 CO$_2$(g)
+ O$_2$(g)

Calculate the total volume of product gases at 2.12atm and 300$^\circ$C from
the detonation of 450g of nitroglycerin.

\vspace{0.25in}

Dalton's Law of Partial Pressure explains that for a mixture of gases, the
total pressure is the sum of the partial pressure of each component. In another
way, the partial pressure of each component can be determined
\begin{equation}
  P_A = \chi_A P_\text{Total} = \frac{n_A}{n_\text{Total}} P_\text{Total}
\end{equation}
where $P_A$ is the partial pressure for a component A, $P_\text{Total}$ is the total
pressure, and $n$ is the amount of moles.

7) Practice: Dalton's Law of Partial Pressures
A flask contains a mixture of 1.25 mols of hydrogen gas and 2.90 moles
of oxygen gas. If the total pressure is 104.kPa, what is the partial
pressure of each gas?
  \vspace{1.5in}

\end{document}

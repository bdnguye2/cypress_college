%%%%%%%%%%%%%%%%%%%%%%%%%%%%%%%%%%%%%%%%%%%%%%%%%%%%%%%%%%%%%%%%%%%%%%%%
% Preamble
%%%%%%%%%%%%%%%%%%%%%%%%%%%%%%%%%%%%%%%%%%%%%%%%%%%%%%%%%%%%%%%%%%%%%%%%
\documentclass[12pt]{article}
%
% Packages and other includes
% Pagination
\usepackage[letterpaper, margin=1in]{geometry}
%
% Graphics, floats, tables
\usepackage{graphicx, color, float, array}
\graphicspath{{image/}}
%
% Fonts
\usepackage[T1]{fontenc} % best for Western European languages
\usepackage{lmodern} % Latin Modern instead of CM
\usepackage{textcomp} % required to get special symbols
%
% Math
\usepackage{amsmath, amssymb}
\usepackage{enumerate}
\usepackage{braket}
% 
% Hyperlinks
\usepackage[colorlinks,linkcolor={red},citecolor={blue},
urlcolor={blue}]{hyperref} 
%
% Definitions and settings
% Paragraph indent and spacing
\setlength{\parskip}{0.4\baselineskip}
\setlength{\parindent}{0in}
%
% Math mode version of "r" column type (requires array package)
\newcolumntype{R}{>{$}r<{$}}
% Title, authors, date
\title{\textbf{Exam 1 Study Guide}}
\date{\today}

\begin{document}

\maketitle 

This is a checklist based on the lecture and textbook materials. It is not
expected to be an all encompassing study guide but rather, provides a guideline for
your studies.

\textbf{Chapter 4: Chemical Composition}

\begin{itemize}
  \setlength\itemsep{0em}
\item Mass percent composition formula
\item The concept of the mol (Avogadro's number)
\item Finding molar masses
\item Molarity (mol/L)
\item Dilutions ($M_1V_1 = M_2V_2$)
\end{itemize}

\textbf{Chapter 5: Chemical Reactions and Equations}

\begin{itemize}
  \setlength\itemsep{0em}
\item Components of chemical reaction - states, reactants, and
  products
\item Balancing chemical equations
\item Classes of chemical reactions - decomposition, combination,
  single- and double-displacement
\item Metal reactivity for single-displacement reactions
\item Acid-base reactions for double-displacement reactions
\item Solubility rules for precipitation reactions (recall: experiment 1
  - filtration technique)
\end{itemize}

\textbf{Chapter 6: Quantities in Chemical Reactions}

\begin{itemize}
  \setlength\itemsep{0em}
\item Meaning of the balanced chemical equation (analogy to cookbook
  recipe)
\item Stoichiometry - mole ratios, converting mols-mols and mass-mass
\item Determining limiting and excess reagents
\item Percent yield, actual yield, theoretical yield
\item Energy changes
  \begin{itemize}
  \item Law of conservation of energy
  \item $q = mC\Delta T$
  \item Endothermic vs Exothermic reactions
  \item Energy sign conventions (+ and - signs)
  \item Calorimetry
  \item Heat changes in chemical reactions
  \end{itemize}
\end{itemize}

\textbf{Chapter 7: Electron Structure of the Atom}

\begin{itemize}
\item Electromagnetic radiation and energy of a photon
  ($E_\text{photon} = \frac{hc}{\lambda}$)
\item Atomic spectra and continuous spectra
\item ROYGBV (700 nm to 400 nm) and corresponding energy
\end{itemize}

\end{document}

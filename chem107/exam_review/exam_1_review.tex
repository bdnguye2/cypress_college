%%%%%%%%%%%%%%%%%%%%%%%%%%%%%%%%%%%%%%%%%%%%%%%%%%%%%%%%%%%%%%%%%%%%%%%%
% Preamble
%%%%%%%%%%%%%%%%%%%%%%%%%%%%%%%%%%%%%%%%%%%%%%%%%%%%%%%%%%%%%%%%%%%%%%%%
\documentclass[12pt]{article}
%
% Packages and other includes
% Pagination
\usepackage[letterpaper, margin=1in]{geometry}
%
% Graphics, floats, tables
\usepackage{graphicx, color, float, array}
\graphicspath{{image/}}
%
% Fonts
\usepackage[T1]{fontenc} % best for Western European languages
\usepackage{lmodern} % Latin Modern instead of CM
\usepackage{textcomp} % required to get special symbols
%
% Math
\usepackage{amsmath, amssymb}
\usepackage{enumerate}
\usepackage{braket}
% 
% Hyperlinks
\usepackage[colorlinks,linkcolor={red},citecolor={blue},
urlcolor={blue}]{hyperref} 
%
% Definitions and settings
% Paragraph indent and spacing
\setlength{\parskip}{0.4\baselineskip}
\setlength{\parindent}{0in}
%
% Math mode version of "r" column type (requires array package)
\newcolumntype{R}{>{$}r<{$}}
% Title, authors, date
\title{\textbf{Exam 1 Study Guide}}
\date{\today}

\begin{document}

\maketitle 

This is a checklist based on the lecture and textbook materials. It is not
expected to be an all encompassing study guide and provides a guideline for
your studies.

\textbf{Chapter 1: Matter and Energy}

\begin{itemize}
  \setlength\itemsep{0em}
\item Classification - pure substance and mixture
\item Different states of matter and its properties - solid, liquid, and gas
\item Physical vs chemical changes
\item Conservation of Energy
\item Conservation of Mass
\item Scientific notation e.g. $164.23 = 1.6423 \times 10^2$
\item[] \textbf{Significant figures}
  \begin{itemize}
  \item What do significant figures imply?
  \item Leading, sandwiched, and trailing zeroes
  \item Rounding rules for multiplying, division, addtion and substraction
  \item Combining multiple steps
  \end{itemize}
\item Unit conversion and prefixes
\item Scientific method and examples where scientific method is applied
\end{itemize}

\textbf{Chapter 2: Atoms, Ions, and the Periodic Table}

\begin{itemize}
  \setlength\itemsep{0em}
\item Dalton's Atomic Theory
\item Law of definite proportions
\item What are atoms made of?
\item Millikan's oil-drop experiment
\item[] \textbf{J.J. Thompson}
  \begin{itemize}
  \item Cathode-ray tube experiment
  \item Plum Pudding Model
  \end{itemize}
\item Isotopes, atomic number, and mass number
\item What are ions?
\item Mass spectrometer
\item Relative atomic mass calculation
\item Periodic Table and its classifications
\end{itemize}

\textbf{Chapter 3: Chemical Compounds}

\begin{itemize}
  \setlength\itemsep{0em}
\item Classifying ionic and molecular compounds
\item Familiarize with the periodic table symbols
  and memorize polyatomic ions
\item Understand the oxidation states for elements
\item Naming rules for ionic and molecular compounds
\item Naming acids
\end{itemize}

\textbf{Chapter 4: Chemical Composition}

\begin{itemize}
  \setlength\itemsep{0em}
\item Mass percent composition formula
\item The concept of the mol (Avogadro's number)
\item Finding molar masses
\item Molarity (mol/L)
\item Dilutions ($M_1V_1 = M_2V_2$)
\end{itemize}

%\textbf{Chapter 5: Chemical Reactions and Equations}
%
%\begin{itemize}
%  \setlength\itemsep{0em}
%\item Components of chemical reaction - states, reactants, and
%  products
%\item Balancing chemical equations
%\item Classes of chemical reactions - decomposition, combination,
%  single- and double-displacement
%\item Solubility rules for precipitation reactions (recall: experiment 1
%  - filtration technique)
%\end{itemize}

\end{document}

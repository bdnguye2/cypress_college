%%%%%%%%%%%%%%%%%%%%%%%%%%%%%%%%%%%%%%%%%%%%%%%%%%%%%%%%%%%%%%%%%%%%%%%%
% Preamble
%%%%%%%%%%%%%%%%%%%%%%%%%%%%%%%%%%%%%%%%%%%%%%%%%%%%%%%%%%%%%%%%%%%%%%%%
\documentclass[12pt]{article}
%
% Packages and other includes
% Pagination
\usepackage[letterpaper, margin=1in]{geometry}

%
% Graphics, floats, tables
\usepackage{graphicx, xcolor, float, array}
\graphicspath{{image/}}
%
% Fonts
\usepackage[T1]{fontenc} % best for Western European languages
\usepackage{lmodern} % Latin Modern instead of CM
\usepackage{textcomp} % required to get special symbols
%
% Math
\usepackage{amsmath, amssymb}
\usepackage{enumerate}
\usepackage{braket}
% 
% Hyperlinks
\usepackage[colorlinks,linkcolor={red},citecolor={blue},
urlcolor={blue}]{hyperref} 
%
% Definitions and settings
% Paragraph indent and spacing
\setlength{\parskip}{0.4\baselineskip}
\setlength{\parindent}{0in}
%
% Math mode version of "r" column type (requires array package)
\newcolumntype{R}{>{$}r<{$}}
\newcommand{\brian}[1]{{\color{orange}{#1}}}
% Title, authors, date
\title{\textbf{Homework 10}}
\date{\vspace{-2em}\today}

\begin{document}

\maketitle 

Weekly homework assignments are posted approximately one week prior to the
due date. Collaborations are encouraged and students must report all collaborators
in writing on each assignment. All external sources (websites, books) must be
properly cited. Additional problems are listed at the end of each assignment.
This week's assignment is due \textit{Friday, Nov 18th at 11:59pm.}

\textbf{Electron Configurations}

1) Determine the electron configurations for the following elements:
(2 pts)

a) V
\vspace{0.2in}

b) Sr
\vspace{0.2in}

c) Ge
\vspace{0.2in}

d) Au
\vspace{0.2in}

e) I
\vspace{0.2in}

2) Determine the electron configurations for the following ions:
(2 pts)

a) Fe$^{3+}$
\vspace{0.2in}

b) Y$^{3+}$
\vspace{0.2in}

c) Pb$^+$
\vspace{0.2in}

d) O$^{2-}$
\vspace{0.2in}

e) Mo$^{5+}$
\vspace{0.4in}

\textbf{Periodic Properties of Atoms}

\textit{Ionization Energy}

3) Predict which as the largest first ionization energy: Mg, Ba, B, O,
and Te. (1 pt)
\vspace{0.4in}

4) Based on their positions in the periodic table, predict which has the
smallest first ionization energy: Na, Cs, N, F, I.  (1 pt)
\vspace{0.4in}

5) Which main group atom would be expected to have the lowest second ionization
energy? (1 pt)
\vspace{0.4in}

6) Based on their positions in the periodic table, rank the following atoms or
compounds in order of increasing first ionization energy: Na, Mg, O, S, Si, and
He. (1 pt)
\vspace{0.4in}

\textit{Atomic Radius}

7) Rank the following based on atomic size: H, Li, B, O, F, At, Ra. (1 pt)
\vspace{0.4in}

8) Based on their positions in the periodic table, list the following ions in order
of increasing radius: K$^+$, Ca$^{2+}$, Al$^{3+}$, Si$^{4+}$, I$^-$. (1 pt)

\vfill

\textbf{Optional Textbook Problems:} Ch. 7- $7.55-7.83$ odd, $7.85-7.107$

\end{document}

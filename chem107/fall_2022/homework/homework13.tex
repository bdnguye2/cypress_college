%%%%%%%%%%%%%%%%%%%%%%%%%%%%%%%%%%%%%%%%%%%%%%%%%%%%%%%%%%%%%%%%%%%%%%%%
% Preamble
%%%%%%%%%%%%%%%%%%%%%%%%%%%%%%%%%%%%%%%%%%%%%%%%%%%%%%%%%%%%%%%%%%%%%%%%
\documentclass[12pt]{article}
%
% Packages and other includes
% Pagination
\usepackage[letterpaper, margin=1in]{geometry}

%
% Graphics, floats, tables
\usepackage{graphicx, xcolor, float, array}
\graphicspath{{image/}}
%
% Fonts
\usepackage[T1]{fontenc} % best for Western European languages
\usepackage{lmodern} % Latin Modern instead of CM
\usepackage{textcomp} % required to get special symbols
%
% Math
\usepackage{amsmath, amssymb}
\usepackage{enumerate}
\usepackage{braket}
% 
% Hyperlinks
\usepackage[colorlinks,linkcolor={red},citecolor={blue},
urlcolor={blue}]{hyperref} 
%
% Definitions and settings
% Paragraph indent and spacing
\setlength{\parskip}{0.4\baselineskip}
\setlength{\parindent}{0in}
%
% Math mode version of "r" column type (requires array package)
\newcolumntype{R}{>{$}r<{$}}
\newcommand{\brian}[1]{{\color{orange}{#1}}}
% Title, authors, date
\title{\textbf{Homework 13}}
\date{\vspace{-2em}\today}

\begin{document}

\maketitle 

Weekly homework assignments are posted approximately one week prior to the
due date. Collaborations are encouraged and students must report all collaborators
in writing on each assignment. All external sources (websites, books) must be
properly cited. Additional problems are listed at the end of each assignment.
This week's assignment is due \textit{Fri, Dec 9th at 11:59pm.}

1) A sample of helium gas at 298.15K and 1.03atm occupies a volume of 5.00L. How
many moles of helium gas are there? (1 pt)

\vspace{1.5in}

2) A scientist collect CO$_2$(g) into a tank that contains a mixture of N$_2$(g) and
He(g). The total pressure is 5.45atm. Vapor pressures of N$_2$(g) and He(g) are 1.20atm
and 2.72atm, respectively. What is the partial pressure of CO$_2$(g)? (1 pt)

\vspace{1.5in}

3) N$_2$(g) is collected over H$_2$O(g) at 40.0$^\circ$C. What is the partial pressure of nitrogen
if the total pressure is 99.42 kPa? The vapor pressure of H$_2$O(g) at 40$^\circ$C is 7.38kPa.
(1 pt)

\vspace{1.5in}

4) A tank contains 78.0g N$_2$(g) and 42.0g Ne at a total pressure of 4.00atm. What
is the partial pressures of N$_2$ and Ne in atm? (1 pt)

\vspace{1.5in}

5) What volume does 5.50mol Ar have at STP? (1 pt)

\vspace{1.5in}

6) What is the density of water vapor 425.15K and 1.50atm? (1 pt)

\vspace{1.5in}

7) Determine the intermolecular forces for the following compounds: CH$_4$, H$_2$O,
CO$_2$, C$_{12}$H$_{26}$, NH$_3$, SO$_2$. (1 pt)

\vspace{1.5in}

8) Rank the boiling point (highest to lowest) of the following compounds: CH$_4$, H$_2$O,
CO$_2$, C$_{12}$H$_{26}$, NH$_3$, SO$_2$. (1 pt)

\vspace{1.5in}

9) Calculate the heat absorbed when 46.0g of ice at $-10.0^\circ$C is converted to liquid
water at $85.0^\circ$C. The specific heat of ice is 2.03 J/(g $^\circ$C), the molar heat
of fusion of ice is 6,010 J/mol, and the specific heat of water is 4.18 J/(g $^\circ$C).
(1 pt)

\vspace{2in}

10) Car air bags inflate based on the decomposition of sodium azide, NaN$_3$(s):

2 NaN$_3$(s) $\rightarrow$ 2 Na(s) + 3 N$_2$(g)

What mass of NaN$_3$ is needed to fill a 3.50L air bag with nitrogen gas at a pressure
of 1.50atm and 300.K?

\vfill

\textbf{Optional Textbook Problems:} Ch. 9- $9.15 - 9.101$ odd; Ch. 10- $10.31 - 10.79$
odd

\end{document}

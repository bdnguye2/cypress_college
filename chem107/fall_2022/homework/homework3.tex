%%%%%%%%%%%%%%%%%%%%%%%%%%%%%%%%%%%%%%%%%%%%%%%%%%%%%%%%%%%%%%%%%%%%%%%%
% Preamble
%%%%%%%%%%%%%%%%%%%%%%%%%%%%%%%%%%%%%%%%%%%%%%%%%%%%%%%%%%%%%%%%%%%%%%%%
\documentclass[12pt]{article}
%
% Packages and other includes
% Pagination
\usepackage[letterpaper, margin=1in]{geometry}

%
% Graphics, floats, tables
\usepackage{graphicx, xcolor, float, array}
\graphicspath{{image/}}
%
% Fonts
\usepackage[T1]{fontenc} % best for Western European languages
\usepackage{lmodern} % Latin Modern instead of CM
\usepackage{textcomp} % required to get special symbols
%
% Math
\usepackage{amsmath, amssymb}
\usepackage{enumerate}
\usepackage{braket}
% 
% Hyperlinks
\usepackage[colorlinks,linkcolor={red},citecolor={blue},
urlcolor={blue}]{hyperref} 
%
% Definitions and settings
% Paragraph indent and spacing
\setlength{\parskip}{0.4\baselineskip}
\setlength{\parindent}{0in}
%
% Math mode version of "r" column type (requires array package)
\newcolumntype{R}{>{$}r<{$}}
\newcommand{\brian}[1]{{\color{orange}{#1}}}
% Title, authors, date
\title{\textbf{Homework 3}}
\date{\vspace{-2em}\today}

\begin{document}

\maketitle 

Weekly homework assignments are posted approximately one week prior to the
due date. Collaborations are encouraged and students must report all collaborators
in writing on each assignment. All external sources (websites, books) must be
properly cited. Additional problems are listed at the end of each assignment.
This week's assignment is due \textit{Friday, Sept 23rd at 11:59pm.}

\textbf{Extra Practice: Naming Compounds}

1) Do the following problems from the textbook: 3.18 to 3.28 (even), 3.30 to
3.56 (even), and 3.62 to 3.90 (even). 8 pts

\vspace{1in}

\textbf{Relative Atomic Mass}

2) Gallium is composed of two naturally occurring isotopes: Ga-69 ($60.11\%$) and
Ga-71. The mass ratio of Ga-71 to Ga-69 is 1.029. What is the mass of Ga-71? (Hint:
Use the periodic table to find the relative atomic mass of Ga.) 2 pts

\vfill

\end{document}

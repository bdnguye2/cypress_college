%%%%%%%%%%%%%%%%%%%%%%%%%%%%%%%%%%%%%%%%%%%%%%%%%%%%%%%%%%%%%%%%%%%%%%%%
% Preamble
%%%%%%%%%%%%%%%%%%%%%%%%%%%%%%%%%%%%%%%%%%%%%%%%%%%%%%%%%%%%%%%%%%%%%%%%
\documentclass[12pt]{article}
%
% Packages and other includes
% Pagination
\usepackage[letterpaper, margin=1in]{geometry}

%
% Graphics, floats, tables
\usepackage{graphicx, xcolor, float, array}
\graphicspath{{image/}}
%
% Fonts
\usepackage[T1]{fontenc} % best for Western European languages
\usepackage{lmodern} % Latin Modern instead of CM
\usepackage{textcomp} % required to get special symbols
%
% Math
\usepackage{amsmath, amssymb}
\usepackage{enumerate}
\usepackage{braket}
% 
% Hyperlinks
\usepackage[colorlinks,linkcolor={red},citecolor={blue},
urlcolor={blue}]{hyperref} 
%
% Definitions and settings
% Paragraph indent and spacing
\setlength{\parskip}{0.4\baselineskip}
\setlength{\parindent}{0in}
%
% Math mode version of "r" column type (requires array package)
\newcolumntype{R}{>{$}r<{$}}
\newcommand{\brian}[1]{{\color{orange}{#1}}}
% Title, authors, date
\title{\textbf{Homework 9}}
\date{\vspace{-2em}\today}

\begin{document}

\maketitle 

Weekly homework assignments are posted approximately one week prior to the
due date. Collaborations are encouraged and students must report all collaborators
in writing on each assignment. All external sources (websites, books) must be
properly cited. Additional problems are listed at the end of each assignment.
This week's assignment is due \textit{Friday, Nov 4th at 11:59pm.}

\textbf{Rydberg Formula}

The Rydberg formula is generalized for the hydrogen atomic spectra. It
can be used to determine the wavelength of the transition between energy levels
$n$ given by
\begin{equation}
  \frac{1}{\lambda} = R_H\Big(\frac{1}{n_f^2}-\frac{1}{n_i^2}\Big)
  \label{eqn:rydberg}
\end{equation}
where $\lambda$ is the wavelength in m, $R_H$ is the Rydberg constant
($1.097 \times 10^7$ m$^{-1}$), and the final and initial energy states $n_f$ and
$n_i$, respectively. Use the formula to answer the following questions of the
hydrogen spectra. Report all values to 3 significant figures.

1) Using Eqn. \ref{eqn:rydberg}, determine the wavelength in nm of the electronic
transition from $n=1$ to $n=3$. Is this wavelength the same for the transition from
$n=3$ to $n=1$? (2 pts)

\vspace{1.75in}

2) For the $n=1$ to $n=3$ transition, what is the energy of a photon required for
this transition? Is the energy being absorbed or released? (2 pts)

\vspace{2in}

3) For the $n=3$ to $n=1$ transition, what is the energy of a photon required for
this transition? Is the energy being absorbed or released? (2 pts)

\vspace{2in}

4) Lines in an atomic emission spectrum are produced when an electron falls from
a higher level to a lower level. For the $n=3$ to $n=1$ transition, what color
of light does this transition emit? Use Fig. 7.8 on page 267 from the Bauer textbook
to help determine the color. (2 pts)

\vspace{2in}

5) Compute the wavelength for the $n=1$ to $n=2$, $n=2$ to $n=3$, and $n=5$ to $n=6$ transitions.
Compare the wavelengths. Draw a conclusion about the energy required for these transitions.
(2 pts) 

\vfill

\textbf{Optional Textbook Problems:} Ch. 7- $7.9 - 7.35$ odd

\end{document}

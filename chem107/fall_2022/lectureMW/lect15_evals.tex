\documentclass[11pt]{beamer}

\usetheme{metropolis}

\usepackage{graphicx}
\usepackage{physics}
\usepackage{adjustbox}
\usepackage{caption}
\usepackage{chemformula}
\usepackage{quoting}
\usepackage[style=chem-angew,backend=bibtex]{biblatex}
\bibliography{references}
%
% Choose how your presentation looks.
%
% For more themes, color themes and font themes, see:
% http://deic.uab.es/~iblanes/beamer_gallery/index_by_theme.html
%
\mode<presentation>
{
  \usetheme{default}      % or try Darmstadt, Madrid, Warsaw, ...
  \usecolortheme{default} % or try albatross, beaver, crane, ...
  \usefonttheme{default}  % or try serif, structurebold, ...
  \setbeamertemplate{navigation symbols}{}
  \setbeamertemplate{caption}[numbered]
  \setbeamerfont{footnote}{size=\tiny}
} 

\usepackage[english]{babel}
\usepackage[utf8]{inputenc}
\graphicspath{{image/}}

\AtBeginSection[]{
\begin{frame}{Outline}
  \tableofcontents[currentsection]
\end{frame}
}

\title{Last UCI TA Evaluations}
\institute{Chemistry Department, Cypress College}
\date{October 31, 2022}

\begin{document}

\begin{frame}
  \titlepage
\end{frame}

\begin{frame}
  ``This instructor asks his students to explain concepts during discussion, which helps
  create conversations about the alternative ways to do the problems and other ways to
  think about them. However, he expects at least some of his students to know calculus
  (which is not a prerequisite) before entering the class, which made understanding
  the chemistry concepts more difficult. Also, he is very strict in grading free-response
  because he looks for key words. If the concept shows full understanding but is missing
  a couple of his predetermined key concepts, he will not give the free response full
  credit, which makes it difficult sometimes to know what he is looking for in an answer.''
\end{frame}

\begin{frame}
  ``Hi Brian, I really appreciate your humility. In discussion when you said that our
  midterm scores were not just a reflection of us as students, they are also a reflection
  of you as a teacher, I really admired that. Thank you for wanting to help us learn and
  improve. I like that you take the time to answer our questions in discussion. Thank
  you for patiently re-explaining things when someone doesn't understand.''
\end{frame}

\begin{frame}
  ``Brian is a really kind TA! He goes out of his way to socialize with everyone despite
  it being on zoom and tries to get everyone included. He is very patient with us and
  always asks us if we have any questions.''
\end{frame}

\begin{frame}
  ``As a teacher, perhaps going a little slower in concepts that are more complex or hard
  to understand might be helpful.''
\end{frame}

\begin{frame}
  ``Um, this might sound weird or unreasonable, but I figured I should mention it since
  this is an anonymous survey. I know that participation during discussion is part of our
  grade for this class, and I completely understand why. However, I have social anxiety and
  it's inexplicably (and honestly ridiculously) difficult for me to volunteer to do
  homework problems. Like, I literally just sit there and think about volunteering and
  my hands get cold and sweaty and my heart starts beating really fast/loudly and I completely
  freeze up. Then eventually, the window of time for volunteers to step up closes
  and I just sit there regretting the fact that I didn't/couldn't volunteer. I really want to
  contribute and also make sure that I get my participation grade, but I don't know how
  to do that on my own terms...
\end{frame}

\begin{frame}
  The social anxiety is regarding the act of volunteering, not
  the act of solving a problem in front of the entire discussion (I know that that's a very
  irrational line of thinking and that those two things are practically the same, but that's
  just how it feels). Therefore, I think it would really help me out if you (Brian) called
  on students more often. There are a couple of students who consistently volunteer
  every class period, and that's totally great and I think they should be allowed to solve
  the problems they volunteer for,...
\end{frame}

\begin{frame}
  but is it possible for you to limit them to working on
  one problem per discussion, even if they got that problem wrong and want to redeem
  themselves? Or maybe you could split up the problems so that one person only has to
  solve one or two parts, thereby allowing everyone else to have more opportunities to
  participate? I know that this request must sound excessive and that I should just suck
  it up and raise my hand or unmute my mic or even type something in the chat, but I'm
  still working through my social anxiety and I think those accommodations would be
  really helpful.''
\end{frame}

\end{document}

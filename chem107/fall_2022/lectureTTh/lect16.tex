\documentclass[11pt]{beamer}

\usetheme{metropolis}

\usepackage{graphicx}
\usepackage{physics}
\usepackage{adjustbox}
\usepackage{caption}
\usepackage{chemformula}
\usepackage{quoting}
\usepackage[style=chem-angew,backend=bibtex]{biblatex}
\bibliography{references}
%
% Choose how your presentation looks.
%
% For more themes, color themes and font themes, see:
% http://deic.uab.es/~iblanes/beamer_gallery/index_by_theme.html
%
\mode<presentation>
{
  \usetheme{default}      % or try Darmstadt, Madrid, Warsaw, ...
  \usecolortheme{default} % or try albatross, beaver, crane, ...
  \usefonttheme{default}  % or try serif, structurebold, ...
  \setbeamertemplate{navigation symbols}{}
  \setbeamertemplate{caption}[numbered]
  \setbeamerfont{footnote}{size=\tiny}
} 

\usepackage[english]{babel}
\usepackage[utf8]{inputenc}
\graphicspath{{../lectureMW/image/}}

\AtBeginSection[]{
\begin{frame}{Outline}
  \tableofcontents[currentsection]
\end{frame}
}

\title{Review Chem Equations}
\institute{Chemistry Department, Cypress College}
\date{November 8, 2022}

\begin{document}

\begin{frame}
  \titlepage
\end{frame}

\begin{frame}{Class Announcements}
  \textbf{Lecture}
  \begin{itemize}
  \item Review Chemical Equations and Limiting Reagent problems
  \item Work in groups and present the exam problems
    (If we finish, everyone receive 2 EC pt in lieu of HW presentations)
  \item Ch 7 - Electromagnetic Radiation
  \item Quiz and Homework assignment released Fri, Nov 4th at 3pm
  \end{itemize}
\end{frame}

\section{Review: Chemical Equation}

\begin{frame}{Meaning of a Chemical Equation}
  \begin{equation}
    aA + bB \rightarrow cC + dD
  \end{equation}
  where $a$, $b$, $c$, and $d$ are coefficients for the reactants $A$/$B$
  and products $C$/$D$
  
  \begin{itemize}
  \item Provides the means to determine how much product is
    produced for a given amount of reactants
  \item Relate to molar masses, number of molecules, amount
    of moles and masses
  \end{itemize}    
\end{frame}

\begin{frame}{Analogy to Recipe Cookbook}
  \begin{center}
    \includegraphics[scale=0.11]{josh_weiss}
  \end{center}

  Analogy: \href{https://www.youtube.com/watch?v=T5SYu8tyKjM}{Cookbook recipe-
    Popeyes Chicken but better}
\end{frame}

\begin{frame}{Photosynthesis}
  \begin{equation}
    6\text{CO$_2$(g)} + 6\text{H$_2$O(l)} \rightarrow \text{C$_6$H$_{12}$O$_6$(s)}
    + 6\text{O$_2$(g)}
  \end{equation}

  \textbf{Properties}
  \begin{itemize}
  \item Balanced chemical equation satisfies the conservation of mass
  \item Coefficients in front of the molecules represent the relative
    moles of reactants and products
  \item \textbf{Q:} How many moles of H$_2$O(l) are needed to react with 12
    moles of CO$_2$(g)?
  \end{itemize}
\end{frame}

\begin{frame}{Practice: Ammonia Production}
  Ammonia is produced using the Haber Bosch process.
  How many moles of ammonia are produced if 2kg of hydrogen are reacted
  with an excess of nitrogen?

  2 N$_2$(g) + 3 H$_2$(g) $\rightarrow$ 2 NH$_3$(g)
  \vspace{1.2in}
\end{frame}

\begin{frame}{Practice: Acid-Base Reaction}
  Suppose you have 50mL of 1.5M HCl(aq) and you attempt to neutralize the
  acid with 1M NaOH(aq). Write the balanced chemical reaction. Determine
  what volume of 1M NaOH(aq) is needed.
  \vspace{1.2in}
\end{frame}

\begin{frame}{Approaching Limiting Reactant Problems}
  \begin{equation}
    \text{R1} + \text{R2} \rightarrow \text{P1}
  \end{equation}
  \begin{itemize}
  \item Given a certain amount of each reagents (R1 and R2)
    to produce P1, determine
    how much the R2 is needed to completely react with R1
  \item Based on that calculated value, determine whether
    there is enough R2 to completely react with R1
  \item If the amount of R2 is less than what is needed,
    then R2 is the limiting
  \item If the amount of R2 is more than what is needed,
    then R2 is the excess
  \end{itemize}
\end{frame}

\begin{frame}{Practice: Limiting Reagent}
  Consider respiration, it is the process of breaking down sugar molecules:

  C$_6$H$_{12}$O$_6$(s) + 6 O$_2$(g) $\rightarrow$ 6 CO$_2$(g) + 6 H$_2$O

  What mass of carbon dioxide forms in the reaction of 25 g of glucose
  (C$_6$H$_{12}$O$_6$) with 40 g of oxygen?
  \vspace{1.2in}
\end{frame}

\end{document}

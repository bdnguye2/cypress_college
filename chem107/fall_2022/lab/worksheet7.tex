%%%%%%%%%%%%%%%%%%%%%%%%%%%%%%%%%%%%%%%%%%%%%%%%%%%%%%%%%%%%%%%%%%%%%%%%
% Preamble
%%%%%%%%%%%%%%%%%%%%%%%%%%%%%%%%%%%%%%%%%%%%%%%%%%%%%%%%%%%%%%%%%%%%%%%%
\documentclass[12pt]{article}
%
% Packages and other includes
% Pagination
\usepackage[letterpaper, margin=1in]{geometry}
%
% Graphics, floats, tables
\usepackage{graphicx, color, float, array}
\graphicspath{{image/}}
%
% Fonts
\usepackage[T1]{fontenc} % best for Western European languages
\usepackage{lmodern} % Latin Modern instead of CM
\usepackage{textcomp} % required to get special symbols
%
% Math
\usepackage{amsmath, amssymb}
\usepackage{enumerate}
\usepackage{braket}
% 
% Hyperlinks
\usepackage[colorlinks,linkcolor={red},citecolor={blue},
urlcolor={blue}]{hyperref} 
%
% Definitions and settings
% Paragraph indent and spacing
\setlength{\parskip}{0.4\baselineskip}
\setlength{\parindent}{0in}
%
% Math mode version of "r" column type (requires array package)
\newcolumntype{R}{>{$}r<{$}}
% Title, authors, date
\title{\textbf{Worksheet 7}}
\date{Oct 5, 2022}

\begin{document}

\maketitle 

\textbf{Add't: Predicting Reactions}

1) Write a balanced chemical equation for the following reactions:

a) solid lithium with liquid water

b) hydrogen gas with bromine gas

c) solid strontium with iodine gas

\textbf{Limiting Reagent}

2) Iron(III) oxide reacts with carbon monoxide according to the unbalanced
equation:

Fe$_2$O$_3$(s) + CO(g) $\rightarrow$ Fe(s) + CO$_2$(g)

A reaction mixture initially contains 22.55g Fe$_2$O$_3$ and 14.78 g CO.
Once the reaction has occurred as completely as possible, what mass in g of
the excess reactant remains? How much CO$_2$(g) in grams is produced.

3) Zinc sulfide reacts with oxygen according to the reaction given by the
unbalanced equation:

ZnS(s) + O$_2$(g) $\rightarrow$ ZnO(s) + SO$_2$(g)

A reaction mixture initially contains 4.2 mol ZnS and 6.8 mol O$_2$. Once
the reaction has occurred as completely as possible, what amount in moles
of the excess reactant remains?

4) Consider the reaction:

2 NO(g) + 5 H$_2$(g) $\rightarrow$ 2 NH$_3$(g) + 2 H$_2$O(g)

A reaction mixture initially contains 5 moles of NO and 10 moles
of H$_2$. Without doing any calculations, determine which set of
amounts best represents the mixture after the reactants have
reacted as completely as possible. Explain your reasoning.

a) 1 mol NO, 0 mol H$_2$, 4 mol NH$_3$, 4 mol H$_2$O

b) 0 mol NO, 1 mol H$_2$, 5 mol NH$_3$, 5 mol H$_2$O

c) 3 mol NO, 5 mol H$_2$, 2 mol NH$_3$, 2 mol H$_2$O

d) 0 mol NO, 0 mol H$_2$, 4 mol NH$_3$, 4 mol H$_2$O

\newpage

\textbf{Percentage Yield}

5) Incomplete combustion of the fuel in a poorly tuned engine can produce
toxic carbon monoxide along with the usual carbon dioxide and water. In a
test of an off-road motorcycle engine, 1.00L of octane (of mass 702 g) is
burned and it was found that 1.84 kg of carbon dioxide is produced.
What is the percentage yield of carbon dioxide?

6) When 24.0 g of potassium nitrate was heated with lead, lead oxide
is formed along with 13.8 g of potassium nitrite was formed. Write the
balanced chemical equation and calculate the percentage yield of possium
nitrite.

7) When limstone, which is principally calcium carbonate is heated,
carbon dioxide and quicklime (CaO) are produced. Write the balanced
chemical equation. If 17.5 g of carbon dioxide is produced from the
thermal decomposition of 42.73g calcium carbonate, what is the percentage
yield of the reaction?

\textbf{Take it together: Limiting Reagents and Yields}

8) Urea (CH$_4$N$_2$O) is a common fertilizer that is synthesized by
the reaction of ammonia (NH$_3$) with carbon dioxide given by the
unbalanced equation:

NH$_3$(aq) + CO$_2$(aq) $\rightarrow$ CH$_4$N$_2$O(aq) + H$_2$O(l)

In an industrial synthesis of urea, a chemis combines 136.4 kg of
ammonia with 211.4 kg of carbon dioxide and obtains 168.4 kg of
urea. Determine the limiting reactant, theoretical yield of urea,
and percent yield for the reaction

9) Many computer chips are manufactured from silicon, which occurs
in nature as SiO$_2$. When SiO$_2$ is heated to melting, it reacts
with solid carbon to form liquid silicon and carbon monoxide gas.
In an industrial preparation of silicon, 155.8 kg of SiO$_2$ reacts
with 78.3 kg of carbon to produce 66.1 kg of silicon. Determine
the limiting reactant, theoretical yield, and percent yield for the
reaction.

10) When aqueous solutions of calcium nitrate and phosphoric acid are
mixed, a white solid precipitates.

a) What is the molecular formula of the solid?

b) How many grams of the solid can be formed from 206 g of calcium
nitrate and 150. g of phosphoric acid?

\end{document}

%%%%%%%%%%%%%%%%%%%%%%%%%%%%%%%%%%%%%%%%%%%%%%%%%%%%%%%%%%%%%%%%%%%%%%%%
% Preamble
%%%%%%%%%%%%%%%%%%%%%%%%%%%%%%%%%%%%%%%%%%%%%%%%%%%%%%%%%%%%%%%%%%%%%%%%
\documentclass[12pt]{article}
%
% Packages and other includes
% Pagination
\usepackage[letterpaper, margin=1in]{geometry}
%
% Graphics, floats, tables
\usepackage{graphicx, color, float, array}
\graphicspath{{image/}}
%
% Fonts
\usepackage[T1]{fontenc} % best for Western European languages
\usepackage{lmodern} % Latin Modern instead of CM
\usepackage{textcomp} % required to get special symbols
%
% Math
\usepackage{amsmath, amssymb}
\usepackage{enumerate}
\usepackage{braket}
% 
% Hyperlinks
\usepackage[colorlinks,linkcolor={red},citecolor={blue},
urlcolor={blue}]{hyperref} 
%
% Definitions and settings
% Paragraph indent and spacing
\setlength{\parskip}{0.4\baselineskip}
\setlength{\parindent}{0in}
%
% Math mode version of "r" column type (requires array package)
\newcolumntype{R}{>{$}r<{$}}
% Title, authors, date
\title{\textbf{Ch 7+8: Periodic Trends and Lewis Structures}}
\date{\today}

\begin{document}

\maketitle 

\textbf{Electron Configurations and Periodic Trends}

1) Write the electron configuration for each ion:
O$^{2-}$, Br$^-$, Sr$^{2+}$, Co$^{3+}$. Cu$^{2+}$, Cl$^-$, P$^{3-}$,
K$^-$, Mo$^{3+}$, and V$^{3+}$

2) Consider these elements: N, Mg, O, F, Al.

a) Write the electron configuration for each element.

b) Arrange the elements in order of decreasing atomic radius.

c) Arrange the elements in order of increasing ionization energy.

d) Use the electron configurations in part a to explain the
differences between your answers to parts b and c.
\vspace{2in}

3) Explain why atomic radius decreases as we move to the right
across a period for main-group elements but not for transition
elements.

\newpage

\textbf{Lewis Structures}

4) Draw the Lewis structure and determing the geometry for the following
compounds: CH$_4$, CH$_3$Cl, CH$_2$Cl$_2$,  CHCl$_3$ and CCl$_4$

a) Determine which bonds are polar.

b) Determine whether the molecule is polar.

\vspace{1.5in}

5) Draw the Lewis structure for acetic acid (CH$_2$COOH)

a) Determine which bonds are polar.

b) Determine whether the molecule is polar.

c) Draw the Lewis structure of the anion acetate (CHCOO$^-$)
and determine the formal charges on the atoms. Include all possible
resonance structures.

\vspace{1.5in}

6) Draw the Lewis structures, classify the geometry, determine which molecules
are polar, and if nonpolar, explain why: CO$_2$, SO$_2$, SO$_3$, BH$_3$, and O$_2$

\end{document}

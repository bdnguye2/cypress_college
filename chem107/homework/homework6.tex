%%%%%%%%%%%%%%%%%%%%%%%%%%%%%%%%%%%%%%%%%%%%%%%%%%%%%%%%%%%%%%%%%%%%%%%%
% Preamble
%%%%%%%%%%%%%%%%%%%%%%%%%%%%%%%%%%%%%%%%%%%%%%%%%%%%%%%%%%%%%%%%%%%%%%%%
\documentclass[12pt]{article}
%
% Packages and other includes
% Pagination
\usepackage[letterpaper, margin=1in]{geometry}

%
% Graphics, floats, tables
\usepackage{graphicx, xcolor, float, array}
\graphicspath{{image/}}
%
% Fonts
\usepackage[T1]{fontenc} % best for Western European languages
\usepackage{lmodern} % Latin Modern instead of CM
\usepackage{textcomp} % required to get special symbols
%
% Math
\usepackage{amsmath, amssymb}
\usepackage{enumerate}
\usepackage{braket}
% 
% Hyperlinks
\usepackage[colorlinks,linkcolor={red},citecolor={blue},
urlcolor={blue}]{hyperref} 
%
% Definitions and settings
% Paragraph indent and spacing
\setlength{\parskip}{0.4\baselineskip}
\setlength{\parindent}{0in}
%
% Math mode version of "r" column type (requires array package)
\newcolumntype{R}{>{$}r<{$}}
\newcommand{\brian}[1]{{\color{orange}{#1}}}
% Title, authors, date
\title{\textbf{Homework 6}}
\date{\vspace{-2em}\today}

\begin{document}

\maketitle 

Weekly homework assignments are posted approximately one week prior to the
due date. Collaborations are encouraged and students must report all collaborators
in writing on each assignment. All external sources (websites, books) must be
properly cited. Additional problems are listed at the end of each assignment.
This week's assignment is due \textit{Friday, Oct 14th at 11:59pm.}

1) The combustion of gasoline produces carbon dioxide and water.
Assume gasoline to be pure octane (C$_8$H$_{18}$) and calculate the
mass (in kg) of carbon dioxide that is added to the atmosphere
per 1.0 kg of octane burned. Report to  (Hint: Begin by writing a balanced
equation for the combustion reaction.) (2 pts)

\vspace{2in}

2) A mixture of C$_3$H$_8$ and C$_2$H$_2$ has a mass of 2.0 g. It is burned in
excess O$_2$ to form a mixture of water and carbon dioxide that
contains 1.5 times as many moles of CO$_2$ as of water. Find the
mass of C$_2$H$_2$ in the original mixture. (2 pts)

\newpage

3) The reaction of NH$_3$ and O$_2$ forms NO and H$_2$O. The NO can be
used to convert P$_4$ to P$_4$O$_6$, forming N$_2$ in the process. The P$_4$O$_6$
can be treated with water to form H$_3$PO$_3$, which forms PH$_3$ and
H$_3$PO$_4$ when heated. Write all chemical equations including states.
Find the mass of PH$_3$ that forms from the reaction of 1.50 g NH3. Report
to 3 significant figures. (3 pts)

\vspace{2.5in}

4) Metallic aluminum reacts with MnO$_2$ at elevated temperatures to
form manganese metal and aluminum oxide. A mixture of the two
reactants is $47.2\%$ mass percent Al. Determine the theoretical yield
(in grams) of manganese from the reaction of 250g of this mixture. Report
to 3 significant figures. (3 pts)

\vfill

\textbf{Optional Textbook Problems:} Ch. 5 - $5.55 - 5.73$ odd; $5.101 - 105$ odd;
5.113, 5.115

\end{document}

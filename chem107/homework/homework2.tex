%%%%%%%%%%%%%%%%%%%%%%%%%%%%%%%%%%%%%%%%%%%%%%%%%%%%%%%%%%%%%%%%%%%%%%%%
% Preamble
%%%%%%%%%%%%%%%%%%%%%%%%%%%%%%%%%%%%%%%%%%%%%%%%%%%%%%%%%%%%%%%%%%%%%%%%
\documentclass[12pt]{article}
%
% Packages and other includes
% Pagination
\usepackage[letterpaper, margin=1in]{geometry}

%
% Graphics, floats, tables
\usepackage{graphicx, xcolor, float, array}
\graphicspath{{image/}}
%
% Fonts
\usepackage[T1]{fontenc} % best for Western European languages
\usepackage{lmodern} % Latin Modern instead of CM
\usepackage{textcomp} % required to get special symbols
%
% Math
\usepackage{amsmath, amssymb}
\usepackage{enumerate}
\usepackage{braket}
% 
% Hyperlinks
\usepackage[colorlinks,linkcolor={red},citecolor={blue},
urlcolor={blue}]{hyperref} 
%
% Definitions and settings
% Paragraph indent and spacing
\setlength{\parskip}{0.4\baselineskip}
\setlength{\parindent}{0in}
%
% Math mode version of "r" column type (requires array package)
\newcolumntype{R}{>{$}r<{$}}
\newcommand{\brian}[1]{{\color{orange}{#1}}}
% Title, authors, date
\title{\textbf{Homework 2}}
\date{\vspace{-2em}\today}

\begin{document}

\maketitle 

Weekly homework assignments are posted approximately one week prior to the
due date. Collaborations are encouraged and students must report all collaborators
in writing on each assignment. All external sources (websites, books) must be
properly cited. Additional problems are listed at the end of each assignment.
This week's assignment is due \textit{Fri, Sept 16th at 11:59pm.}

\textbf{The Atom and Isotopes}

1) True or false. If false, change the statement to make it true. (1 pt)

\begin{itemize}
\item If an atom has an unequal number of protons and electrons, it will be
  charge-neutral.
\item Electrons are attracted to protons.
\item Protons and electrons have charges of the same magnitude but opposite
  sign.
\item Some atoms don't have any protons.
\item Protons have twice the mass of neutrons.
\item Electrons are much heavier than neutrons and protons.
\end{itemize}

\vspace{1in}

\textbf{Conservation of Mass}

2) A volatile liquid (one that readily evaporates) is put into a jar, and the
jar is then sealed. Does the mass of the sealed jar and its content change
upon the vaporization of the liquid? (1 pt)

\pagebreak

\textbf{Atomic Mass}

3) An element has four naturally occurring isotopes with the masses and natural
abundances given here. Find the atomic mass of the element and identify it. (1 pt)
\begin{table}[hbpt]
  \begin{tabular}{ccc}
    Isotope & Mass (amu) & Abundance ($\%$) \\
    \hline
    1 & 135.90714 & 0.19 \\
    2 & 137.90599 & 0.25 \\
    3 & 139.90543 & 88.43\\
    4 & 141.90924 & 11.11
  \end{tabular}
\end{table}

\vspace{1in}

4) Gallium has two naturally occurring isotopes with the following masses and
natural abundances: (1 pt)
\begin{table}[H]
  \begin{tabular}{ccc}
    Isotope & Mass (amu) & Abundance ($\%$) \\
    \hline
    Ga-69 & 68.92558 & 60.108 \\
    Ga-71 & 70.92470 & 39.892
  \end{tabular}
\end{table}

\vspace{1in}

5) Silicon has three naturally occurring isotopes (Si-28, Si-29, and Si-30).
The mass and natural abundance of Si-28 are 27.9769 amu and 92.2$\%$, respectively.
The mass and natural abundance of Si-29 are 28.9765 amu and 4.67$\%$, respectively.
Find the mass and natural abundance of Si-30. (1 pt)


\vfill

\textbf{Optional Textbook Problems:} Ch. 2- $2.31-2.79$ odd, $2.145-2.153$ odd

\end{document}

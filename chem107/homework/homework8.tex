%%%%%%%%%%%%%%%%%%%%%%%%%%%%%%%%%%%%%%%%%%%%%%%%%%%%%%%%%%%%%%%%%%%%%%%%
% Preamble
%%%%%%%%%%%%%%%%%%%%%%%%%%%%%%%%%%%%%%%%%%%%%%%%%%%%%%%%%%%%%%%%%%%%%%%%
\documentclass[12pt]{article}
%
% Packages and other includes
% Pagination
\usepackage[letterpaper, margin=1in]{geometry}

%
% Graphics, floats, tables
\usepackage{graphicx, xcolor, float, array}
\graphicspath{{image/}}
%
% Fonts
\usepackage[T1]{fontenc} % best for Western European languages
\usepackage{lmodern} % Latin Modern instead of CM
\usepackage{textcomp} % required to get special symbols
%
% Math
\usepackage{amsmath, amssymb}
\usepackage{enumerate}
\usepackage{braket}
% 
% Hyperlinks
\usepackage[colorlinks,linkcolor={red},citecolor={blue},
urlcolor={blue}]{hyperref} 
%
% Definitions and settings
% Paragraph indent and spacing
\setlength{\parskip}{0.4\baselineskip}
\setlength{\parindent}{0in}
%
% Math mode version of "r" column type (requires array package)
\newcolumntype{R}{>{$}r<{$}}
\newcommand{\brian}[1]{{\color{orange}{#1}}}
% Title, authors, date
\title{\textbf{Homework 8}}
\date{\vspace{-2em}\today}

\begin{document}

\maketitle 

Weekly homework assignments are posted approximately one week prior to the
due date. Collaborations are encouraged and students must report all collaborators
in writing on each assignment. All external sources (websites, books) must be
properly cited. Additional problems are listed at the end of each assignment.
This week's assignment is due \textit{Friday, Oct 28th at 11:59pm.}

\textbf{Electromagnetic Radiation}

1) Calculate the wavelength of each frequency of electromagnetic radiation reporting
to 4 significant figures (2pts):

a) 100.2 MHz (typical frequency for FM radio broadcasting) \\
b) 1070. kHz (typical frequency for AM radio broadcasting) \\
c) 835.6 MHz (common frequency used for cell phone communication) \\
d) 250.0 $\mu$m (usual frequency used for IR spectroscopy)

\vspace{2in}

2) A laser pulse with wavelength 532 nm contains 4.0 mJ of energy. How many photons
are in the laser pulse? Report to 2 significant figures. (2 pts)

\vspace{2in}

\textbf{Identifying Specific Heat of a Metal}

3) A 75.0 g piece of metal that had been submerged in boiling water reaching thermal
equilibrium. Quickly after equilibrium, the unknown metal is transferred into 100.0 g
of water initially at 15.0 $^\circ$C reaching the final temperature of 26.8 $^\circ$C.
What is the specific heat capcity of the metal? Identify the metal from Table 6.2 on
pg 239 of the textbook. The specific heat of water is 4.184 J/(g $^\circ$C). Report to 3
significant figures. (3 pts)

\vspace{2.5in}

\textbf{Heat from Chemical Reaction}

4) Lithium metal spontaneously reacts in the presence of water to form lithium
hydroxide and hydrogen gas. The reaction releases 403.26 kJ/mol of heat. (3 pts)

a) Write the balanced chemical equation.

b) Determine the amount of heat released when 1.00 g of Li metal is added to water.

\vfill

\textbf{Optional Textbook Problems:} Ch. 6- $6.99, 6.100$, Ch. 7 $7.5-7.25$ odd

\end{document}

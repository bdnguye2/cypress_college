\documentclass[11pt]{beamer}

\usetheme{metropolis}

\usepackage{graphicx}
\usepackage{physics}
\usepackage{adjustbox}
\usepackage{caption}
\usepackage{chemformula}
\usepackage{quoting}
\usepackage[style=chem-angew,backend=bibtex]{biblatex}
\bibliography{references}
%
% Choose how your presentation looks.
%
% For more themes, color themes and font themes, see:
% http://deic.uab.es/~iblanes/beamer_gallery/index_by_theme.html
%
\mode<presentation>
{
  \usetheme{default}      % or try Darmstadt, Madrid, Warsaw, ...
  \usecolortheme{default} % or try albatross, beaver, crane, ...
  \usefonttheme{default}  % or try serif, structurebold, ...
  \setbeamertemplate{navigation symbols}{}
  \setbeamertemplate{caption}[numbered]
  \setbeamerfont{footnote}{size=\tiny}
} 

\usepackage[english]{babel}
\usepackage[utf8]{inputenc}
\graphicspath{{image/}}

\AtBeginSection[]{
\begin{frame}{Outline}
  \tableofcontents[currentsection]
\end{frame}
}

\title{Chapter 8: Chemical Bonding}
\institute{Chemistry Department, Cypress College}
\date{Nov 9, 2022}

\begin{document}

\begin{frame}
  \titlepage
\end{frame}

\begin{frame}{Class Announcements}
  \textbf{Lab}
  \begin{itemize}
  \item Experiment 17 Lewis Structures and Molecular Models
  \item Basic steps for lewis structures
  \item Reminder - Need $70\%$ of laborator points to pass
    the course
  \end{itemize}

  \textbf{Lecture}
  \begin{itemize}
  \item Finish up Ch 7 and begin Ch 8
  \item Go over homework 9 (EC for students who present)
  \item Quiz and Homework assignment released Fri, Nov 11th at 3pm
  \end{itemize}
\end{frame}

\section{Review: }

\begin{frame}{Principles for Filling Atomic Orbitals}
  \textbf{Aufbau principle} - electrons fill an orbital starting with
  the lowest energy level

  \textbf{Pauli exclusion princple} - No two electrons with the same
  spin can occupy the same orbital

  \textbf{Hund's Rule} - Maximize the number of unpaired electrons
\end{frame}

\begin{frame}{Electron Configurations of Ions}
  \textbf{Cations} - Remove electrons from the highest energy atomic
  orbitals

  \textbf{Anions} - Follow the same Aufbau principle by filling orbitals
  with the lowest energy level

  \onslide<2->{\textbf{Q:} For transition metals, which atomic orbitals, s or d,
    do you begin removing electrons from?
  }
\end{frame}

\section{Types of Bonds}

\begin{frame}{Water is Life}
  \centering
  \includegraphics[width=0.4\linewidth]{water}
  \includegraphics[width=0.4\linewidth,trim={0 6in 7in 0},clip]{molec_example}

  \begin{itemize}
  \item Liquid water made up of moles upon moles of water molecules
  \item Molecules are made up of atoms connected by ``chemical bonds''
  \end{itemize}
\end{frame}

\begin{frame}{What are Chemical Bonds?}
  \begin{center}
    \includegraphics[width=0.8\linewidth]{single_elect_orb}
  \end{center}
  \textbf{Bonds are made up of atomic orbitals}
  \begin{itemize}
  \item Overlap of atomic orbitals lead to the formation of molecular
    orbitals (same energy and specific orientation)
  \end{itemize}
\end{frame}

\begin{frame}{Example of p-orbitals}
  \centering
  \includegraphics[width=\linewidth]{p_sigma}
  \begin{itemize}
  \item Depending on the orientation, p-orbitals
    will form a bond
  \end{itemize}
\end{frame}

\subsection{Ionic and Covalent Bonds}

\begin{frame}{Ionic Bonds}
  \begin{center}
    \includegraphics[width=0.6\linewidth]{nacl}
  \end{center}
  
  \textbf{Ionic Compounds} - Made up of cation and anion

  \textbf{Ionic Bonds} - Hold the cations and anions together;
  purely electrostatic interaction

  \onslide<2->{\textbf{Q:} For ionic bond, are the electrons shared
    between the cation and anion?}
\end{frame}

\begin{frame}{Covalent Bonds}
  \begin{center}
    \includegraphics[width=0.6\linewidth]{molec_example}
  \end{center}
  \begin{itemize}
  \item Electrons are shared between atoms to achieve
    the octet rule
  \item \textit{Note:} Octet rule can be broken for atoms
    after the 3rd row e.g. P, S, Cl, etc.
  \end{itemize}
\end{frame}

\subsection{Electronegativity}

\begin{frame}{Electronegativity: Tug-of-War}
  \centering
  \includegraphics[width=0.8\linewidth]{water_tug}
  \begin{itemize}
  \item Sharing of electrons can lead to unequal pull
    (electronegativity)
  \end{itemize}
\end{frame}

\begin{frame}{Electronegativity Trends}
  \centering
  \includegraphics[width=\linewidth]{electronegativity}
\end{frame}

\begin{frame}{Practice: Polarity}
  \textbf{Which of the following is the most polar bond?}

    C--C; C--H; N--H; O--H; F--H; Se--H
\end{frame}

\section{Drawing Lewis Structures}

\section{VSEPR Theory}

\end{document}

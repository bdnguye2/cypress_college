\documentclass[11pt]{beamer}

\usetheme{metropolis}

\usepackage{graphicx}
\usepackage{physics}
\usepackage{adjustbox}
\usepackage{caption}
\usepackage{chemformula}
\usepackage{quoting}
\usepackage[style=chem-angew,backend=bibtex]{biblatex}
\bibliography{references}
%
% Choose how your presentation looks.
%
% For more themes, color themes and font themes, see:
% http://deic.uab.es/~iblanes/beamer_gallery/index_by_theme.html
%
\mode<presentation>
{
  \usetheme{default}      % or try Darmstadt, Madrid, Warsaw, ...
  \usecolortheme{default} % or try albatross, beaver, crane, ...
  \usefonttheme{default}  % or try serif, structurebold, ...
  \setbeamertemplate{navigation symbols}{}
  \setbeamertemplate{caption}[numbered]
  \setbeamerfont{footnote}{size=\tiny}
} 

\usepackage[english]{babel}
\usepackage[utf8]{inputenc}
\graphicspath{{../lectureMW/image/}}

\AtBeginSection[]{
\begin{frame}{Outline}
  \tableofcontents[currentsection]
\end{frame}
}

\title{Chapter 2: Atoms, Ions, and the Periodic Table}
\institute{Chemistry Department, Cypress College}
\date{Sept 8, 2022}

\begin{document}

\begin{frame}
  \titlepage
\end{frame}

\begin{frame}{Class Announcements}
  \begin{itemize}
  \item When uploading assignments, be certain that
    the file is in a readable format e.g. docx, png, jpeg,
    and pdf
  \item This week only, any late HW assignments will not be
    penalized $50\%$; submit late assignments by the Sept 8th
    at 11:59pm
  \item Quiz \#2 released this Fri, Sept 9 at 11am and due Tues,
    Sept 12 at 11am
  \item Homework \#2 released this Fri, Sept 9 at 11am and due
    Fri, Sept 16 at 11am
  \end{itemize}  
\end{frame}

\begin{frame}{Lecture Weekly Agenda}

  \begin{itemize}
  \item Finished Ch 2 - pg $56 - 88$
  \item In-class Ch 2 worksheet
  \item Time permits, begin Ch 3
  \end{itemize}
\end{frame}

\section{Review: Scientific Notation and Unit Conversion}

\begin{frame}{Scientific Notation}
  \begin{align*}
    2.78 \times 10^{-8} & - 5.689 \times 10^{-9} = \\
    \frac{7.18 - 6.729}{2.51\times 7.343} & = \\
    \frac{7.9\times 10^{34}}{8.235 \times 10^{23}} & =
  \end{align*}
\end{frame}

\begin{frame}{Unit Conversion}
  \textbf{Volume Conversion}

  $9.2 \text{m}^3$ to mm$^3$

  $581.74$ mL to m$^3$

  $0.53$ g/cm$^3$ to kg/m$^3$
\end{frame}

\section{Review: Relative Atomic Mass}

\begin{frame}{Experiment: Mass Spectroscopy}
  \begin{center}
    \includegraphics[width=\linewidth]{mass_spect}
  \end{center}

  \begin{itemize}
  \item Ionizes the atom and electric field accelerates atoms
  \item Time of flight - heavier atoms will travel slower
    than lighter ones
  \item Weighted average of atomic masses
  \end{itemize}  
\end{frame}

\begin{frame}{Relative Atomic Mass}
  \begin{equation}
    \text{Relative Atomic Mass} = (I_1\times A_1) + (I_2\times A_2) + \dots
  \end{equation}
  where $I$ is the mass of the isotope, and $A$ is the
  relative abundance between 0 and 1
\end{frame}

\begin{frame}{Hydrogen Isotopes and Applications}
  \begin{center}
    \includegraphics[width=0.75\linewidth]{hydro_iso}
  \end{center}

  \begin{itemize}
  \item Hydrogen ($^1_1$H), deuterium ($^2_1$D), and tritium ($^3_1$T)
    have relative abundances of $99.84\%$, $0.0156\%$, and trace amounts,
    respectively
  \item \textbf{Q:} Which hydrogen isostope is the highest in abundance?
  \end{itemize}
\end{frame}

\begin{frame}{Hydrogen Isostopes and Applications}
  \begin{center}
    \includegraphics[width=0.75\linewidth]{hydro_iso}
  \end{center}

  \textbf{Applications}
  
  \begin{itemize}
  \item Semiconductor production enhancing Si-H bond by preventing
    chemical erosion and Hot Carrier Effect
  \item Chemical labeling to track chemical reactions
  \item Medicinal chemistry - FDA approved the first deuterium-labeled
    drug (\href{https://pubs.acs.org/doi/10.1021/acs.jmedchem.8b01808}{reference})
  \end{itemize}
\end{frame}

\section{Periodic Table - Grouped Elements}

\begin{frame}{Review: Modern Period Table}
  \centering
  \includegraphics[width=\linewidth]{ptable}
\end{frame}

\begin{frame}{Alkali Metal}
  \begin{center}
    \includegraphics[scale=0.2]{alkali_metal}
  \end{center}
  
  \begin{itemize}
  \item Lower densities than other metals
  \item Extremely soft metals
  \item Highly reactive e.g. forming H$_2$ when in
    contact with water
  \item Prefer to lose an electron
  \end{itemize}
\end{frame}

\begin{frame}{Alkaline Earth Metal}
  \begin{center}
    \includegraphics[width=0.4\linewidth]{alkaline_metal}
  \end{center}
  
  \begin{itemize}
  \item Fairly reactive metals
  \item Can form solutions with a pH greater than
    7 (more basic or alkaline)
  \item Calcium and magnesium important for life
  \item Prefer to lose 2 electrons
  \end{itemize}
\end{frame}

\begin{frame}{Transition Metals}
  \begin{center}
    \includegraphics[scale=0.12]{copper_pan}
  \end{center}
  
  \begin{itemize}
  \item Easily malleable and great conductors of heat and
    electricity
  \item High melting points except mercury (liquid at Room
    temperature)
  \item High densities
  \item Oxidation states (ability to gain/lose electrons) can
    vary between 1+ to 6+
  \end{itemize}
\end{frame}

\begin{frame}{Actinides and Lanthanides}
  \begin{center}
    \includegraphics[scale=0.3]{quantum_comp}
  \end{center}

  \begin{itemize}
  \item Radioactive due to instability
  \item Silvery/silvery-white luster in metallic form
  \item Potential application to quantum computers and
    nuclear power
  \item Oxidation states can range from 2+ to 7+
  \end{itemize}
\end{frame}

\begin{frame}{Materials for Quantum Computing: Lanthanide Complexes}
  \begin{columns}
    \column{0.4\textwidth}
    \centering
    \includegraphics[scale=0.15]{gd_homo}
    \textbf{[1-Gd]$^{-1}$} HOMO
    
    \column{0.7\textwidth}
    \centering
    \includegraphics[scale=0.14]{gd_uv}
  \end{columns}

  \begin{itemize}
  \item Understanding the electronic structure
  \item Hysteresis - electronic spin memory
  \item \href{https://pubs.acs.org/doi/10.1021/jacs.1c03098}
    {Lanthanide MoS$_4$ research article}
  \end{itemize}
\end{frame}

\begin{frame}{Nuclear Power Plants}
  \begin{center}
    \includegraphics[width=\linewidth]{nuclear_plant}
  \end{center}
\end{frame}

\begin{frame}{Halogens}
  \begin{itemize}
  \item Fairly toxic and form acids when combined with
    hydrogen
  \item Readily react with metals to form salts e.g.
    NaCl
  \item Important for drug development due to their ``sticky''
    nature
  \item Prefers to gain an electron
  \end{itemize}
\end{frame}

\begin{frame}{Cancer Therapeutics}
  \begin{center}
    \includegraphics[scale=0.17]{sea_squirt.jpg}
  \end{center}

  \begin{itemize}
  \item Chlorolissoclimide is a potent cancer drug that
    is naturally found in sea squirts
  \item Understanding the structure--activity relationships
    e.g. interactions between drug and ribosome
  \end{itemize}
\end{frame}

\begin{frame}{My Research Project: Chlorolissoclimide}
  \begin{center}
    \includegraphics[trim={0 4.2in 0 0},clip,scale=0.4]{lisso_drug}
  \end{center}

  \begin{itemize}
  \item \href{https://www.nature.com/articles/nchem.2800}
    {Chlorolissoclimide research article}
  \end{itemize}
\end{frame}

\begin{frame}{Noble Gases}
  \begin{center}
    \includegraphics[scale=0.7]{neon_lights}
  \end{center}
  
  \begin{itemize}
  \item Colorless, odorless, tasteless, and non-flammable
    under standard conditions
  \item Extremely non-reactive and most stable elements
  \item Do not like to gain or lose electrons
  \end{itemize}
\end{frame}

\begin{frame}{Practice: Periodic Table}
  Group the elements into the following groups
  \begin{itemize}
  \item Br
  \item K
  \item Mg
  \item Al
  \item Mn
  \item Ar
  \item U
  \end{itemize}
\end{frame}

\begin{frame}{Practice}
  What is the charge of the ions for each of the following elements?

  \begin{itemize}
  \item Al
  \item P
  \item Br
  \item S
  \end{itemize}
\end{frame}

\end{document}
